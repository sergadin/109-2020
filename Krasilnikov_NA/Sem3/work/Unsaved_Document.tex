\documentclass[11pt]{article}
\usepackage[utf8]{inputenc}
\usepackage[english, russian]{babel}
    \title{\textbf{База данных "Нагрузка"}}
    \date{{}}
    
    \addtolength{\topmargin}{-3cm}
    \addtolength{\textheight}{3cm}
\begin{document}

\maketitle
\thispagestyle{empty}
\section{Структура проекта}
Проект состоит из трех каталогов: server, client, test.
В каталоге server соджержится исходный код для сервера, в client скрипт для запуска клиента, а в test находятся наборы тестовых данных и скрипт для тестирования.
\section{Сборка и запуск}
Для сборки сервера необходимо зайти в каталог server и выполнить команду make. Для удаления созданных объектных файлов используется команда make clean.\\\\
Запуск сервера осуществляется с помощью команды:\\\textbf{python3 server.py [host ip address] [port]}\\\\ (запуск производится в каталоге server) Оба аргумента командной строки обязательны. Для завершения работы сервера необходимо использовать Ctrl+C.\\\\
Запуск клиента производится через команду:\\\textbf{python3 client.py [server ip address] [port]}\\\\
(запуск производится в каталоге client).

\section{Запуск тестов}
	Для запуска тестов необходимо зайти в каталог test. И в коммандной строке ввести:\\\textbf{python3 tester.py 5 testfile}\\\\Изначально в папке содержится 6 файлов с запросами, которые отправляют клиенты (файлы 0.test - 5.test) и файл \textbf{testfile} в котором содержится сгенерированная автоматически база данных, которую загружает первый клиент(ему соответствует файл с именем 0.txt). Первый аргумент скрипта tester.py - количество клиентов, которые делают запросы к базе данных(нулевой клиент, загружающий базу данных из файла, не считается), второй параметр - файл с загружаемой базой данных.\\\\В процессе тестирования запускается 5 клиентов и один сервер. Когда все клиенты отработают, сервер завершает работу, также создаются файлы, в которые клиенты записывают результат запроса к базе (файлы с именами 0.test\textunderscore{}res - 5.test\textunderscore{}res)

\section{Описание команд используемых БД}
Все команды, которые подаются на вход базе данных являются \textbf{однострочными}.\\\\
\textbf{addCourse [название курса] [описание нагрузок(возможно с пробельными символами)]} - добавление нового курса. \textbf{Каждому курсу присваивается уникальный номер, нумерация с 0}. Описание нагрузок в строгом формате - после каждого наименования учебной нагрузки следует целое число - кол-во часов для данного вида нагрузки. Имена нагрузок изначально не определены, поэтому пользователь может добавлять различные виды нагрузок в базу данных. (Пример: addCourse course1 lectures 10 hw 15).\\\\
\textbf{deleteCourse index/name [index/name]} - удаление курса по индексу или по названию курса. Если после наименования команды идет ключевое слово index, то после слова index должно идти целое число - номер нужного курса, а после ключевого слова name - строка - имя нужного курса. Нумерация курсов начинается с 0.\\(Пример 1: deleteCourse index 0; Пример 2: deleteCourse name course1) \\\\
\textbf{updateCourse index/name [index/name] [описание нагрузок(возможно с пробельными символами)]} - изменение нагрузки на определенном курсе (описание курса аналогично описанию курса при использовании deleteCourse).\\\\
\textbf{updateCourseName [index] [новое имя курса]} - изменение имени курса по индексу курса.\\\\
\textbf{addWorker [название факультета] [название кафедры] [должность] [имя работника]} - добавление нового работника. Все аргументы обязательны и должны идти в строгом порядке (Названия должностей, факультетов и должностей не определены изначально, что дает возможность пользователю брать любые названия должностей, кафедр и факультетов. \textbf{Каждое название - одно слово без пробельных символов}).\\\\
\textbf{uploadFromTxtFile [имя текстового файла]} - загрузка базы данных из файла (Хранение данных производится в виде команд для БД).\\\\
\textbf{saveAsTxtFile [имя текстового файла]} - сохранение базы данных в файл.\\\\
\textbf{deleteWorker [факультет] [кафедра] [должность] [имя]} - удаление сотрудника из базы данных. Для большей гибкости добавлена возможность удаления нескольких сотрудников с помощью символа !. Если один из 4х обязательных параметров неопределен, то можно указать вместо него ! и тогда операция будет применена ко всем работникам, которые соответствуют указанному шаблону. (Пример: в базе было 2 сотрудника: \textbf{fac1 kaf1 post1 name1} и \textbf{fac1 kaf2 post2 name2}. Если указать команду \textbf{deleteWorker fac1 kaf2 post2 name2} или команду \textbf{deleteWorker fac1 kaf2 ! !}, то будет удален второй работник, а первый останется в базе, а если применить команду \textbf{deleteWorker ! ! ! !}, то будут удалены оба сотрудника).\\\\
\textbf{deleteWorkersCourse [факультет] [кафедра] [должность] [имя] [номер курса]} - удаление курса у сотрудника соответствующего шаблону (аналогичного шаблону в команде deleteWorker).\\\\
\textbf{updateWorkersCourse [факультет] [кафедра] [должность] [имя] [номер курса] [строка (возможно с пробельными символами)]} - изменение курса у сотрудника соответствующего шаблону (аналогичного шаблону в команде deleteWorker).\\\\
\textbf{addWorkersCourse [факультет] [кафедра] [должность] [имя] [номер курса]} - добавление нового курса  сотруднику, который соответствует шаблону (аналогичного шаблону в команде deleteWorker).\\\\
\textbf{getWorkersCourse [факультет] [кафедра] [должность] [имя]} - вывод на консоль всех курсов сотрудника, который соответствует шаблону (аналогичного шаблону в команде deleteWorker).\\\\
\textbf{select columns [названия в иерархии сотрудников(только faculty, department, post) или наименования нагрузок(например: lectures, hw)] end\textunderscore{}columns rows\textunderscore{}for\textunderscore{}all [наименования в иерархии(например: professor)] sum\textunderscore{}row/end\textunderscore{}row} - выборка из базы данных по параметрам, которая отображается в виде таблицы в текстовом формате. После ключевого слова columns необходимо указать те критерии, которые будут отображаться по столбцам. Поле имени указывать не нужно, оно добавляется автоматически. Отбор осуществляется по принципу объединения имен иерархии указанных в поле после rows\textunderscore{}for\textunderscore{}all. sum\textunderscore{}row означает, что необходимо в конце таблицы вывести дополнительно строку, в которой указаны просуммированные данные по столбцам (если столбец содержит информацию в виде чисел), end\textunderscore{}row значит, что суммирование по столбцам производиться не будет.
\\\\\\Пример: \textbf{select columns faculty lectures end\textunderscore{}columns rows\textunderscore{}for\textunderscore{}all docent professor sum\textunderscore{}row} - Выведет таблицу из 3х столбцов (в первом имя сотрудника, во втором название факультета, в третьем сумму часов лекций, которые проводит этот сотрудник.) В каждой строке будет указан один сотрудник, а в последней строке будет указано суммарное кол-во часов лекций у всех сотрудников из выборки.\\\\
\textbf{exit} - команда для завершенмя работы клиента (он отключается от сервера).
\end{document}

